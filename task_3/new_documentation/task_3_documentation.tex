\documentclass[a4paper,10pt,ngerman]{scrartcl}
\usepackage{babel}
\usepackage[T1]{fontenc}
\usepackage[utf8x]{inputenc}
\usepackage[a4paper,margin=2.5cm,footskip=0.5cm]{geometry}

% todo: check TeilnahmeId
\newcommand{\Aufgabe}{Aufgabe 3: Eisbudendilemma}
\newcommand{\TeilnahmeId}{56860}
\newcommand{\Name}{Christopher Besch}


% header and footer
\usepackage{scrlayer-scrpage, lastpage}
\setkomafont{pageheadfoot}{\large\textrm}
\lohead{\Aufgabe}
\rohead{Teilnahme-ID: \TeilnahmeId}
\cfoot*{\thepage{}/\pageref{LastPage}}

% title position
\usepackage{titling}
\setlength{\droptitle}{-1.0cm}

% for math commands and symbols
\usepackage{amsmath}
\usepackage{amssymb}

% for images
\usepackage{graphicx}

% for tables
\usepackage{tabularx}

% for algorithms
\usepackage{algpseudocode}

% for source code
\usepackage{listings}
\usepackage{color}
\definecolor{mygreen}{rgb}{0,0.6,0}
\definecolor{mygray}{rgb}{0.5,0.5,0.5}
\definecolor{mymauve}{rgb}{0.58,0,0.82}
\lstset{
  keywordstyle=\color{blue},commentstyle=\color{mygreen},
  stringstyle=\color{mymauve},rulecolor=\color{black},
  basicstyle=\footnotesize\ttfamily,numberstyle=\tiny\color{mygray},
  captionpos=b, % sets the caption-position to bottom
  keepspaces=true, % keeps spaces in text
  numbers=left, numbersep=5pt, showspaces=false,showstringspaces=true,
  showtabs=false, stepnumber=2, tabsize=2, title=\lstname
}
\lstdefinelanguage{JavaScript}{ % JavaScript is the only non-predefined language
  keywords={break, case, catch, continue, debugger, default, delete, do, else, finally, for, function, if, in, instanceof, new, return, switch, this, throw, try, typeof, var, void, while, with},
  morecomment=[l]{//},
  morecomment=[s]{/*}{*/},
  morestring=[b]',
  morestring=[b]",
  sensitive=true
}

% these packages must be loaded last
\usepackage{hyperref}
\usepackage{cleveref}

% c++ source code setup
\lstset{
    language=C++,
    basicstyle=\small\sffamily,
    numbers=left,
    numberstyle=\tiny,
    frame=tb,
    tabsize=4,
    columns=fixed,
    showstringspaces=false,
    showtabs=false,
    keepspaces,
    commentstyle=\color{red},
    keywordstyle=\color{blue}
}

% title
\title{\textbf{\Huge\Aufgabe}}
\author{\LARGE Teilnahme-ID: \LARGE \TeilnahmeId \\\\
	    \LARGE Bearbeiter/-in dieser Aufgabe: \\ 
	    \LARGE \Name\\\\}
\date{\LARGE\today}

\begin{document}

\maketitle
\tableofcontents

\vspace{0.5cm}

\section{Lösungsidee}
% Die Idee der Lösung sollte hieraus vollkommen ersichtlich werden, ohne dass auf die eigentliche Implementierung Bezug genommen wird.
Das Ziel ist es, ein Arrangement bestehend aus drei Positionen für Eisdielen zu generieren, das in einer Abstimmung durch kein anderes Arrangement abgelöst werden kann.
Diese Arrangements werden stabil genannt.
Hierzu darf die Eisbudendistanz, die Strecke zwischen einem beliebigen Haus und der nächsten Eisbude, von nicht mehr als der Hälfte der Hauser durch ein anderes Arrangement verkürzt werden.
Wäre dies der Fall, würde die Ablösung mehr Ja- als Nein-Stimmen erhalten.
Hieraus geht hervor, dass für eine optimale Lösung für alle möglichen Arrangements alle Arrangements überprüft werden müssen.
Das Arrangement, das auf Stabilität getestet wird, wird Test-Arrangement genannt, das, mit dem überprüft wird, Check-Arrangement.
Es lässt sich leicht erkennen, dass ein derartiger Algorithmus mit einer Laufzeit von $O(n^6)$ nicht verwendbar ist.

Als Versuch der Optimierung werden bevor sie getestet werden, alle Arrangement sortiert.
Hierzu wird für jedes mögliche Arrangement ein Score berechnet.
Dieser entspricht der durchschnittlichen Eisbudendistanz aller Häuser.
Nun stellt sich heraus, dass die stabilen Arrangement einen niedrigen Score aufweisen.
Dies lässt sich damit erklären, dass je kleiner die Eisbudendistanz eines Hauses in einem Test-Arrangement ist, desto weniger Check-Arrangement existieren, die eine noch geringere Eisbudendistanz für das Haus generieren.
Wenn die Eisbudendistanz beispielsweise $0$ beträgt existiert kein einziges Check-Arrangement, dem dieses Haus eine Ja-Stimme geben würde, da ein Haus bei gleichbleibender Eisbudendistanz immer gegen einen Wechsel stimmt.
Wenn die Eisbudendistanz ein den maximalen Wert, dem halben Umfang des Sees, entspricht, stimmt es für alle Check-Arrangement, abgesehen von denen, die die Eisbudendistanz nicht verändern.

Die durchschnittliche Eisbudendistanz lässt sich dementsprechend als \glqq Zufriedenheitsgrad\grqq{} des Dorfes interpretieren.
Je höher er ist, desto unwahrscheinlicher wird eine Veränderung durchgesetzt.

Allerdings muss dieser Wert nicht zwangsweise mit der Stabilität eines Arrangements übereinstimmen.

\section{Umsetzung}
% Hier wird kurz erläutert, wie die Lösungsidee im Programm tatsächlich umgesetzt wurde. Hier können auch Implementierungsdetails erwähnt werden.
Die Lösungsidee wird in C++ implementiert.

\paragraph{Einlese der Eingabedatei}
Als erster Schritt wird in der Funktion \textbf{read\_file} die Eingabedatei gelesen.
Hierbei wird überprüft, ob die Eingabedatei dem gegebenen Format entspricht, wenn nicht wird das Programm abgebrochen.
Hierzu wird ein Makro \textbf{raise\_error()} verwendet, dass die Ausführung des Programmes abbricht und eine möglichst informative Fehlermeldung zurückgibt.
Dieses Makro wird ebenfalls für alle Methoden aller Klassen verwendet, um z.b. Segmentation Faults zu verhindern.

Es wird die Menge der Adressen aller Häuser in einem std::vector<int> und der Umfang des Sees in einem int gespeichert.


\section{Beispiele}
% Genügend Beispiele einbinden! Die Beispiele von der BwInf-Webseite sollten hier diskutiert werden, aber auch eigene Beispiele sind sehr gut – besonders wenn sie Spezialfälle abdecken. Aber bitte nicht 30 Seiten Programmausgabe hier einfügen!
Nun wird das Programm mit allen Beispieldateien ausgeführt.

\paragraph{eisbuden1.txt}
Mit der Eingabe

20 7

0 2 3 8 12 14 15

gibt das Programm aus, dass es keine stabilen Positionen gibt.

\paragraph{eisbuden2.txt}
Mit der Eingabe

50 15

3 6 7 9 24 27 36 37 38 39 40 45 46 48 49

gibt das Programm die Menge an stabilen Positionen aus:

45

\paragraph{eisbuden3.txt}
Mit der Eingabe

50 16

2 7 9 12 13 15 17 23 24 35 38 42 44 45 48 49

gibt das Programm die Menge an stabilen Positionen aus:

2, 3, 4, 5, 6 und 7

\paragraph{eisbuden4.txt}
Mit der Eingabe

100 19

6 12 23 25 26 28 31 34 36 40 41 52 66 67 71 75 80 91 92

gibt das Programm die Menge an stabilen Positionen aus:

34

\paragraph{eisbuden5.txt}
Mit der Eingabe

247 24

2 5 37 43 72 74 83 87 93 97 101 110 121 124 126 136 150 161 185 200 201 230 234 241

gibt das Programm die Menge an stabilen Positionen aus:

93, 94, 95, 96 und 97

\paragraph{eisbuden6.txt}
Mit der Eingabe

437 36

4 12 17 23 58 61 67 76 93 103 145 154 166 170 192 194 209 213 221 225 239 250 281 299 312 323 337 353 383 385 388 395 405 407 412 429

gibt das Programm aus, dass es keine stabilen Positionen gibt.

\paragraph{eisbuden7.txt}

\subsection{Eigene Beispiele}

\paragraph{myeisbuden0.txt}
Mit der Eingabe

12 4

0 3 6 9

gibt das Programm die Menge an stabilen Positionen aus:

0, 1, 2, 3, 4, 5, 6, 7, 8, 9, 10 und 11

Dies ergibt Sinn, da die Häuser gleichmäßig verteilt sind, womit nie eine Mehrheit gegen jegliche Positionen gefunden werden kann.

\paragraph{myeisbuden0.txt}
Mit der Eingabe

10 0


0, 1, 2, 3, 4, 5, 6, 7, 8 und 9

Dies ergibt Sinn, da keine Häuser vorhanden sind, die für eine Umlegung stimmen könnten.

\section{Quellcode}
% Unwichtige Teile des Programms sollen hier nicht abgedruckt werden. Dieser Teil sollte nicht mehr als 2–3 Seiten umfassen, maximal 10.
Dies sind die wichtigsten Funktionen:
\begin{lstlisting}
int get_distance(int circumference, int place_a, int place_b)
{
    int direct_distance = std::abs(place_a - place_b);
    // take shortest way, direct or the othe rdirection
    return std::min(direct_distance, circumference - direct_distance);
}

bool vote(int circumference, int house_place, int old_place, int new_place)
{
    // is new place better?
    if (get_distance(circumference, house_place, new_place) <
    get_distance(circumference, house_place, old_place))
        return true;
    return false;
}

bool is_stable(int circumference, std::vector<int> &houses, int test_place)
{
    // would any other place win an election against test_place?
    for (int other_place = 0; other_place < circumference; other_place++)
    {
        int trues = 0;
        for (int house : houses)
            if (vote(circumference, house, test_place, other_place))
                trues++;

        if (trues > houses.size() - trues)
            return false;
    }
    return true;
}

std::vector<int> get_stabel_places(int circumference, std::vector<int> &houses)
{
    // go thorugh all possible places
    std::vector<int> result;
    for (int test_place = 0; test_place < circumference; test_place++)
        if (is_stable(circumference, houses, test_place))
            result.push_back(test_place);
    return result;
}
\end{lstlisting}

\end{document}
