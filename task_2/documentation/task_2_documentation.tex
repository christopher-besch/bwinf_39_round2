\documentclass[a4paper,10pt,ngerman]{scrartcl}
\usepackage{babel}
\usepackage[T1]{fontenc}
\usepackage[utf8x]{inputenc}
\usepackage[a4paper,margin=2.5cm,footskip=0.5cm]{geometry}

% todo: check TeilnahmeId
\newcommand{\Aufgabe}{Aufgabe 2: Spießgesellen}
\newcommand{\TeilnahmeId}{56860}
\newcommand{\Name}{Christopher Besch}


% header and footer
\usepackage{scrlayer-scrpage, lastpage}
\setkomafont{pageheadfoot}{\large\textrm}
\lohead{\Aufgabe}
\rohead{Teilnahme-ID: \TeilnahmeId}
\cfoot*{\thepage{}/\pageref{LastPage}}

% title position
\usepackage{titling}
\setlength{\droptitle}{-1.0cm}

% for math commands and symbols
\usepackage{amsmath}
\usepackage{amssymb}

% for images
\usepackage{graphicx}

% for tables
\usepackage{tabularx}
\usepackage{booktabs}

% for algorithms
\usepackage{algpseudocode}

% for source code
\usepackage{listings}
\usepackage{color}
\definecolor{mygreen}{rgb}{0,0.6,0}
\definecolor{mygray}{rgb}{0.5,0.5,0.5}
\definecolor{mymauve}{rgb}{0.58,0,0.82}
\lstset{
  keywordstyle=\color{blue},commentstyle=\color{mygreen},
  stringstyle=\color{mymauve},rulecolor=\color{black},
  basicstyle=\footnotesize\ttfamily,numberstyle=\tiny\color{mygray},
  captionpos=b, % sets the caption-position to bottom
  keepspaces=true, % keeps spaces in text
  numbers=left, numbersep=5pt, showspaces=false,showstringspaces=true,
  showtabs=false, stepnumber=2, tabsize=2, title=\lstname
}
\lstdefinelanguage{JavaScript}{ % JavaScript is the only non-predefined language
  keywords={break, case, catch, continue, debugger, default, delete, do, else, finally, for, function, if, in, instanceof, new, return, switch, this, throw, try, typeof, var, void, while, with},
  morecomment=[l]{//},
  morecomment=[s]{/*}{*/},
  morestring=[b]',
  morestring=[b]",
  sensitive=true
}

% these packages must be loaded last
\usepackage{hyperref}
\usepackage{cleveref}

% c++ source code setup
\lstset{
    language=C++,
    basicstyle=\small\sffamily,
    numbers=left,
    numberstyle=\tiny,
    frame=tb,
    tabsize=4,
    columns=fixed,
    showstringspaces=false,
    showtabs=false,
    keepspaces,
    commentstyle=\color{red},
    keywordstyle=\color{blue}
}

% title
\title{\textbf{\Huge\Aufgabe}}
\author{\LARGE Teilnahme-ID: \LARGE \TeilnahmeId \\\\
	    \LARGE Bearbeiter/-in dieser Aufgabe: \\ 
	    \LARGE \Name\\\\}
\date{\LARGE\today}

\begin{document}

\maketitle
\tableofcontents

\vspace{0.5cm}

\section{Lösungsidee}
% Die Idee der Lösung sollte hieraus vollkommen ersichtlich werden, ohne dass auf die eigentliche Implementierung Bezug genommen wird.
Die von Donald beobachteten Spieße lassen sich in \autoref{tab:alle_Spiesse} darstellen.

\begin{table}[h!t]
    \begin{center}
        \begin{tabular}{l|l}
            \textbf{Schüsseln} & \textbf{Früchte}            \\
            \midrule
            1, 4, 5            & Apfel, Banane, Brombeere    \\
            3, 5, 6            & Banane, Pflaume, Weintraube \\
            1, 2, 4            & Apfel, Brombeere, Erdbeere  \\
            2, 6               & Erdbeere, Pflaume
        \end{tabular}
        \caption{Auflistung aller Spieße}
        \label{tab:alle_Spiesse}
    \end{center}
\end{table}

\subsection{Bestimmung der legalen Früchte}
Nun wird in \autoref{tab:alle_Schüsseln} jede Schüssel mit den Früchten der Spieße, die diese Schüssel verwenden, aufgelistet (eine Spalte pro Spieß).
Zudem werden in einer weiteren Spalte alle Früchte notiert, die in Spießen vorkommen, die nicht diese Schüssel verwenden, sie werden verbotene Früchte genannt.
Diese Früchte können nicht von dieser Schüssel sein.

\begin{table}[h!t]
    \begin{center}
        \begin{tabularx}{\linewidth}{l|X|X|X}
            \textbf{Schüssel} & \multicolumn{2}{c|}{\textbf{Spieße von dieser Schüssel}} & \textbf{Verbotene Früchte}                                                  \\
            \midrule
            1                 & Apfel, Banane, Brombeere                                 & Apfel, Brombeere, Erdbeere  & Banane, Weintraube, Erdbeere, Pflaume         \\
            \midrule
            2                 & Apfel, Brombeere, Erdbeere                               & Erdbeere, Pflaume           & Apfel, Brombeere, Banane, Weintraube, Pflaume \\
            \midrule
            3                 & Banane, Pflaume, Weintraube                              &                             & Banane, Apfel, Brombeere, Erdbeere, Pflaume   \\
            \midrule
            4                 & Apfel, Banane, Brombeere                                 & Apfel, Brombeere, Erdbeere  & Banane, Weintraube, Erdbeere, Pflaume         \\
            \midrule
            5                 & Apfel, Banane, Brombeere                                 & Banane, Pflaume, Weintraube & Apfel, Brombeere, Erdbeere, Pflaume           \\
            \midrule
            6                 & Banane, Pflaume, Weintraube                              & Erdbeere, Pflaume           & Banane, Apfel, Brombeere, Erdbeere
        \end{tabularx}
        \caption{Auflistung aller Schüsseln}
        \label{tab:alle_Schüsseln}
    \end{center}
\end{table}

Nun werden für jede Schüssel alle möglicherweise vorliegenden Früchte gesucht; diese werden im Folgenden \glqq legale Früchte\grqq{} genannt.
In jeder Schüssel liegt eine Frucht aus der Menge der legalen Früchte für diese Schüssel.

Alle legalen Früchte für eine bestimmte Schüssel müssen zwei Voraussetzungen erfüllen:
\begin{enumerate}
    \item Die Frucht muss in allen Spießen vorkommen, die diese Schüssel benutzen.
    \item Die Frucht darf für diese Schüssel nicht verboten sein.
\end{enumerate}
Alle legalen Früchte werden in \autoref{tab:alle_Schüsseln_legal} markiert.

\begin{table}[h!t]
    \begin{center}
        \begin{tabularx}{\linewidth}{l|X|X|X}
            \textbf{Schüssel} & \multicolumn{2}{c|}{\textbf{Spieße von dieser Schüssel}} & \textbf{Verbotene Früchte}                                                                         \\
            \midrule
            1                 & \underline{Apfel}, Banane, \underline{Brombeere}         & \underline{Apfel}, \underline{Brombeere}, Erdbeere & Banane, Weintraube, Erdbeere, Pflaume         \\
            \midrule
            2                 & Apfel, Brombeere, \underline{Erdbeere}                   & \underline{Erdbeere}, Pflaume                      & Apfel, Brombeere, Banane, Weintraube, Pflaume \\
            \midrule
            3                 & Banane, Pflaume, \underline{Weintraube}                  &                                                    & Banane, Apfel, Brombeere, Erdbeere, Pflaume   \\
            \midrule
            4                 & \underline{Apfel}, Banane, \underline{Brombeere}         & \underline{Apfel}, \underline{Brombeere}, Erdbeere & Banane, Weintraube, Erdbeere, Pflaume         \\
            \midrule
            5                 & Apfel, \underline{Banane}, Brombeere                     & \underline{Banane}, Pflaume, Weintraube            & Apfel, Brombeere, Erdbeere, Pflaume           \\
            \midrule
            6                 & Banane, \underline{Pflaume}, Weintraube                  & Erdbeere, \underline{Pflaume}                      & Banane, Apfel, Brombeere, Erdbeere
        \end{tabularx}
        \caption{Markierung der legalen Früchte}
        \label{tab:alle_Schüsseln_legal}
    \end{center}
\end{table}


\subsection{Auswahl der Schalen für Donald}
Die folgenden Früchte sind (von Donald) gefordert: Weintraube, Brombeere und Apfel

Es werden alle Schüsseln durchgegangen.
\begin{itemize}
    \item Wenn alle legalen Früchte einer Schüssel gefordert sind, wird diese Schüssel ausgewählt.
          Da es irrelevant ist, welche legale Frucht tatsächlich in der Schüssel liegende ist, liefert die Schüssel eine geforderte Frucht.
    \item Wenn keine der legalen Früchte gefordert sind, ist dies kein ausgewählter Schüssel.
    \item Wenn einige, aber nicht alle legalen Früchte gefordert sind, ist nicht genügend Information vorhanden, da nicht gesagt werden kann, ob die Schüssel eine ausgewählte Frucht enthält oder nicht.
\end{itemize}
Es ergibt sich die Menge aller ausgewählten Schüsseln: 1, 3 und 4

Die Schüsseln 1 und 2 liefern jeweils entweder Äpfel oder Brombeeren.
Da beide Früchte gefordert sind, ist diese Ungewissheit irrelevant.

Mit dieser Methode kann in allen Fällen eine Aussage gefällt werden, ob genügend Information vorliegt und, wenn dies der Fall ist, die Menge der ausgewählten Schüsseln angegeben werden.

\section{Umsetzung}
% Hier wird kurz erläutert, wie die Lösungsidee im Programm tatsächlich umgesetzt wurde. Hier können auch Implementierungsdetails erwähnt werden.
Die Lösungsidee wird in C++ implementiert.
Der einzige Command Line Parameter, den das Programm benötigt, ist der Pfad der Eingabedatei.

\subsection{Einlese der Eingabedatei}
Als erster Schritt wird in der Funktion \textbf{read\_file} die Eingabedatei gelesen.
Hierbei wird überprüft, ob die Eingabedatei dem gegebenen Format entspricht, wenn nicht, wird das Programm abgebrochen.
Hierzu wird ein Makro \textbf{raise\_error()} verwendet, das die Ausführung des Programmes abbricht und eine möglichst informative Fehlermeldung zurückgibt.

Die Fruchtnamen werden mithilfe einer Instanz der Klasse \textbf{LookupTable} aufsteigend und anfangend bei $0$ in Ganzzahlen, Frucht IDs, übersetzt.
Dadurch ist die Anzahl an verschieden Früchten nicht auf das Alphabet limitiert.
Ein weiterer Vorteil ist, dass da die ID aufsteigend und bei $0$ anfangend verteilt werden, die IDs als Indizes in Bit Fields verwendet werden können.
Hierzu wird ein std::vector<bool> verwendet, da die Anzahl der Früchte zum Zeitpunkt der Kompilierung nicht bekannt ist.
So liegt für jede Frucht ein Bit/Bool vor, ist sie $true$, ist diese Frucht enthalten, ist sie $false$, ist sie nicht enthalten.
Diese kompakte Datenstruktur ist ressourcenschonender im Vergleich zu einem std::vector<std::string> mit den Namen aller enthaltenden Früchten.
Zudem können effiziente OR und AND Operationen verwendet werden, um zwei Bit Fields miteinander zu vergleichen, oder weitere Früchte hinzuzufügen.

Die Namen für die Schüsseln werden auf die gleiche Weise behandelt.
Zwar ist angegeben, dass alle Schüsseln nummeriert sind, es ist allerdings nicht klar, wie die Nummern verteilt werden, daher ergibt es Sinn, die Namen der Schüsseln als Strings zu behandeln und ebenfalls eine Instanz von \textbf{LookupTable} zu verwenden.
So können auch für die Schüsseln Bit Fields verwendet werden.
Ein zusätzlicher Vorteil ist, dass so jegliche Namen für die Schüsseln erlaubt sind, sie sind nicht auf Nummern begrenzt.

Als Größe für die Bit Fields wird die angegebene Anzahl an Früchten verwendet.

\medskip
Es werden Instanzen der Klasse \textbf{Skewer} erstellt, diese enthalten die Menge der Früchte auf einem Spieß und die Menge der Schüsseln, von denen die Früchte genommen sind.
Hierbei werden, wie bereits erwähnt, Bit Fields verwendet.
An dieser Stelle werden fehlende Schüsseln und fehlende Früchte hinzugefügt, sodass die Menge an Schüsseln und Früchten der angegebenen Anzahl an Früchten entspricht.
Diese neuen Früchte und Schüsseln erhalten Namen wie \glqq new\_bowl0\grqq{} oder \glqq new\_fruit0\grqq{}.
Es wird überprüft, ob bereits eine Frucht mit dem Namen \glqq new\_fruit0\grqq{} vorliegt, ist dies der Fall wird \glqq new\_fruit1\grqq{} überprüft.
Dies wird so lange ausgeführt, bis ein nicht belegter Name gefunden wurde.
Gleiches gilt für neue Schüsselnamen.

\subsection{Bestimmung der Legalen Früchte}
In der Funktion \textbf{determine\_legal\_fruits} werden die legalen Früchte bestimmt, wie bereits in der Lösungsidee beschrieben.

Zuerst werden dazu für jede Schale \textbf{Bowl} Instanzen erstellt.
Diese enthalten
\begin{itemize}
    \item \textbf{allowed\_fruit\_sets}, ein std::vector mit einem Bit Field pro Spieß, der diese Schale verwendet,
    \item alle verbotenen Früchte (\textbf{disallowed\_fruits}),
    \item alle legalen Früchte (\textbf{legal\_fruits}) und
    \item zwei bools, \textbf{selected} und \textbf{possible\_selected}, auf die später eingegangen wird.
\end{itemize}
Es werden alle Spieße durchgegangen.
Wenn der Spieß diese Schale verwendet, werden dessen Früchte zu \textbf{allowed\_fruit\_sets} hinzugefügt, wenn nicht zu \textbf{disallowed\_fruits}.

Nun werden die legalen Früchte für jede Schale bestimmt.
Dafür werden für jede Frucht die beiden in der Lösungsidee genannten Voraussetzungen überprüft.
Hierzu werden die erlaubten und verbotenen Früchte in std::vector<bool> geladen und miteinenader verglichen.

\subsection{Auswahl der Schalen}
Im letzten Schritt werden alle Schalen durchgegangen.
Hierbei wird überprüft, wie viele der legalen Früchte einer Schale gefordert sind.
Sind alle gefordert, wird \textbf{selected} auf $true$ gesetzt, sind einige aber nicht alle gefordert, wird \textbf{possible\_selected} auf $true$ und \textbf{possible} auf $false$ gesetzt.

Wenn \textbf{possible} nach Durchlauf aller Schalen $true$ bleibt, ist genügend Information vorhanden, um eine Aussage treffen zu können.

\subsection{\textbf{main}}
In der \text{main} Funktion werden die drei genannten Funktionen ausgeführt und die Ergebnisse in der Konsole ausgegeben.

\section{Beispiele}
% Genügend Beispiele einbinden! Die Beispiele von der BwInf-Webseite sollten hier diskutiert werden, aber auch eigene Beispiele sind sehr gut – besonders wenn sie Spezialfälle abdecken. Aber bitte nicht 30 Seiten Programmausgabe hier einfügen!
Nun wird das Programm mit allen Beispieldateien ausgeführt.

\paragraph{spiesse0.txt}
Dies ist das Beispiel, das in dem Aufgabenblatt vorgestellt ist.

Das Programm liefert die richtige Antwort, es ist genügend Information vorhanden und die ausgewählten Schalen sind 1, 4 und 3.

\paragraph{spiesse1.txt}
Das Programm erkennt, dass genügend Information vorliegt und wählt die Schalen 2, 1, 4, 5 und new\_bowl0 aus.

Hier lässt sich erkennen, dass eine Schale in keinem Spieß vorkommt und einen Namen zugeteilt bekommen hat.

\paragraph{spiesse2.txt}
Erneut wird das korrekte Ergebnis produziert, es liegt genügend Information vor und die Schalen 6, 1, 7, 11, 5 und 10 wurden ausgewählt.

\paragraph{spiesse3.txt}
In diesem Fall liegt nicht genügend Information vor, um eindeutig sagen zu können, welche Schalen auszuwählen sind---das System ist unterbestimmt.
Dies gibt das Programm korrekt aus und gibt an, dass die Schalen 7, 10, 12, 5, 1 und 8 definitiv geforderte Früchte enthalten und dass entweder die Schale 2 oder 11 eine geforderte Frucht enthält, es liegt allerdings nicht genügend Information vor, um genau bestimmen zu können, welche das ist.

In der ersten Zeile der Beispieldatei wird die Anzahl der Früchte angegeben, in diesem Fall 15.
Allerdings sind nur 14 verschiedene Früchte in den Spießen und den geforderten Früchten vorzufinden.
Daher wird die fehlende Frucht new\_fruit0 genannt, auch wenn sie für die Lösung des Problems irrelevant ist, da sie nicht gefordert wurde.

\paragraph{spiesse4.txt}
Erneut können die Schalen mit den geforderten Früchten genau bestimmt werden, 7, 9, 2, 6, 8, 12, 13 und 14.

\paragraph{spiesse5.txt}
Genau wie bei \textbf{spiesse3.txt} ist die angegebene Anzahl an Früchten zu hoch, weshalb eine Frucht new\_fruit0 genannt wird.
Das Programm liefert die korrekte Ausgabe, es gibt eine Lösung und die ausgewählten Schalen sind 16, 14, 6, 9, 2, 3, 1, 19, 5, 10, 12, 4 und 20.

\paragraph{spiesse6.txt}
Es liegt genügend Information vor und die ausgewählten Schalen sind 18, 15, 20, 11, 4, 7, 10 und 6.

\paragraph{spiesse7.txt}
Es liegt nicht genügend Information vor und die Schalen 14, 8, 5, 24, 23, new\_bowl0, new\_bowl1 und new\_bowl2 liefern garantiert geforderte Früchte, wobei new\_bowl0, new\_bowl1 und new\_bowl2 Schalen sind, die für keinen der angegebenen Spieße verwendet werden.
Zusätzlich gibt das Programm aus, dass vier der folgenden Schalen geforderte Früchte enthalten: 3, 20, 26, 10, 25 und 18

\subsection{Eigene Beispiele}

\paragraph{myspiesse0.txt}
In der Aufgabenstellung ist angegeben, dass die Namen der Früchte immer unterschiedlichen Anfangsbuchstaben haben.
Dieses Programm limitiert die Eingabedatei allerdings nicht auf diese Weise.

Dies ist in diesem Beispiel gezeigt, es ist eine Kopie von spiesse0.txt, wobei alle Früchte den selben Anfangsbuchstaben haben.
Das Programm produziert das korrekte Ergebnis, Schüsseln 1, 4 und 3.

\paragraph{myspiesse1.txt}
Es handelt sich erneut um eine Kopie von spiesse0.txt.
Hier ist die angegebene Anzahl an Früchten deutlich höher (10) als die in den Spießen und geforderten Früchten gelisteten Schalen und Früchten (6).
Das Programm erstellt daher vier neue Früchte und Schalen.
Es produziert das korrekte Ergebnis: 1, 4 und 3

\paragraph{myspiesse2.txt}
Die Datei sieht wie folgt aus:

\textbf{1}

\textbf{Weintraube}

\textbf{0}

Es ist nur eine Frucht vorhanden, die ebenfalls gefordert wird.
Es liegt keinerlei Information über jegliche Spieße vor.

Das Programm produziert das korrekte Ergebnis, new\_bowl0, da nur eine Schale vorliegt und diese damit die geforderte Frucht enthalten muss.

\paragraph{myspiesse3.txt}
Die Datei sieht wie folgt aus:

\textbf{2}

\textbf{Weintraube}

\textbf{0}

Der Unterschied zu dem vorherigen Beispiel ist die Absolutzahl der Früchte.
In Diesem Fall sind zwei Früchte vorhanden.

Das Programm erstellt eine neue Frucht, da nur eine gefunden werden kann, Weintraube, und zwei neue Schalen.
In diesem Fall ist nicht klar, in welcher Schale die geforderten Weintrauben liegen.
Dies gibt das Programm korrekt aus.

\paragraph{myspiesse4.txt}
Dies ist eine leere Datei, weshalb das Programm mit einer Fehlermeldung abbricht: \glqq File Parsing Error: Not enough elements found!\grqq{}

\paragraph{myspiesse5.txt}
Dies ist eine Kopie von spiesse7.txt, wobei die Nummern aller Schalen durch zufällige Wörter ersetzt wurden.
Das Programm produziert problemlos das korrekte Ergebnis.

\paragraph{myspiesse6.txt}
In diesem Fall wird für die Anzahl aller Früchte \glqq fdfhasdfgdfhsdfh\grqq{} eingesetzt.
Das Programm bricht mit einem Fehler ab: \glqq File Parsing Error: Can't convert "fdfhasdfgdfhsdfh" to int!\grqq{}

\section{Quellcode}
% Unwichtige Teile des Programms sollen hier nicht abgedruckt werden. Dieser Teil sollte nicht mehr als 2–3 Seiten umfassen, maximal 10.
Dies ist der Teil der Funktion \textbf{determine\_legal\_fruits}, der die legalen Früchte für alle Schüsseln auswählt.
\glqq fruit\_look\_up.get\_amount()\grqq{} gibt die Anzahl aller Früchte zurück.
\begin{lstlisting}
for (Bowl &bowl : bowls)
{
    for (int fruit_idx = 0; fruit_idx < fruit_look_up.get_amount(); fruit_idx++)
    {
        // also true when no skewers given <- no info means everything is possible
        bool in_all = true;
        for (const std::vector<bool> &fruit_set : bowl.get_allowed_fruit_sets())
            if (!fruit_set[fruit_idx])
            {
                in_all = false;
                break;
            }

        // for this to be a legal fruit,
        // it has to be in every allowed fruit set and not a disallowed fruit
        if (!bowl.get_disallowed_fruits()[fruit_idx] && in_all)
            bowl.add_legal_fruit(fruit_idx);
    }
}
\end{lstlisting}

Die Funktion \textbf{get\_selected\_bowls} verwendet den folgenden Teil, um die jeweils legalen Früchte für alle Schalen auszuwählen.
\begin{lstlisting}
possible = true;
// test each bowl: is this one providing one of the requested fruits?
for (Bowl &bowl : bowls)
{
    bool all_legal_are_requested = true;
    bool all_legal_are_not_requested = true;
    // go through all the fruits, one of which this bowl provides
    for (int idx = 0; idx < fruit_look_up.get_amount(); idx++)
    {
        if (bowl.get_legal_fruits()[idx])
        {
            if (requested_fruits[idx])
                all_legal_are_not_requested = false;
            else
                all_legal_are_requested = false;
        }
    }

    // when some of this bowls legal fruits are requested and others aren't,
    // there isn't enough information
    if (!all_legal_are_requested && !all_legal_are_not_requested)
    {
        possible = false;
        bowl.set_possibly_selected();
    }
    else if (all_legal_are_requested)
        bowl.set_selected();
}
\end{lstlisting}

\end{document}
