\documentclass[a4paper,10pt,ngerman]{scrartcl}
\usepackage{babel}
\usepackage[T1]{fontenc}
\usepackage[utf8x]{inputenc}
\usepackage[a4paper,margin=2.5cm,footskip=0.5cm]{geometry}

% todo: check TeilnahmeId
\newcommand{\Aufgabe}{Aufgabe 2: Flohmarkt in Langdorf}
\newcommand{\TeilnahmeId}{56860}
\newcommand{\Name}{Christopher Besch}


% header and footer
\usepackage{scrlayer-scrpage, lastpage}
\setkomafont{pageheadfoot}{\large\textrm}
\lohead{\Aufgabe}
\rohead{Teilnahme-ID: \TeilnahmeId}
\cfoot*{\thepage{}/\pageref{LastPage}}

% title position
\usepackage{titling}
\setlength{\droptitle}{-1.0cm}

% for math commands and symbols
\usepackage{amsmath}
\usepackage{amssymb}

% for images
\usepackage{graphicx}

% for tables
\usepackage{tabularx}

% for algorithms
\usepackage{algpseudocode}

% for source code
\usepackage{listings}
\usepackage{color}
\definecolor{mygreen}{rgb}{0,0.6,0}
\definecolor{mygray}{rgb}{0.5,0.5,0.5}
\definecolor{mymauve}{rgb}{0.58,0,0.82}
\lstset{
  keywordstyle=\color{blue},commentstyle=\color{mygreen},
  stringstyle=\color{mymauve},rulecolor=\color{black},
  basicstyle=\footnotesize\ttfamily,numberstyle=\tiny\color{mygray},
  captionpos=b, % sets the caption-position to bottom
  keepspaces=true, % keeps spaces in text
  numbers=left, numbersep=5pt, showspaces=false,showstringspaces=true,
  showtabs=false, stepnumber=2, tabsize=2, title=\lstname
}
\lstdefinelanguage{JavaScript}{ % JavaScript is the only non-predefined language
  keywords={break, case, catch, continue, debugger, default, delete, do, else, finally, for, function, if, in, instanceof, new, return, switch, this, throw, try, typeof, var, void, while, with},
  morecomment=[l]{//},
  morecomment=[s]{/*}{*/},
  morestring=[b]',
  morestring=[b]",
  sensitive=true
}

% these packages must be loaded last
\usepackage{hyperref}
\usepackage{cleveref}

% c++ source code setup
\lstset{
    language=C++,
    basicstyle=\small\sffamily,
    numbers=left,
    numberstyle=\tiny,
    frame=tb,
    tabsize=4,
    columns=fixed,
    showstringspaces=false,
    showtabs=false,
    keepspaces,
    commentstyle=\color{red},
    keywordstyle=\color{blue}
}

% title
\title{\textbf{\Huge\Aufgabe}}
\author{\LARGE Teilnahme-ID: \LARGE \TeilnahmeId \\\\
	    \LARGE Bearbeiter/-in dieser Aufgabe: \\ 
	    \LARGE \Name\\\\}
\date{\LARGE\today}

\begin{document}

\maketitle
\tableofcontents

\vspace{0.5cm}

\section{Lösungsidee}
% Die Idee der Lösung sollte hieraus vollkommen ersichtlich werden, ohne dass auf die eigentliche Implementierung Bezug genommen wird.

Die von Donald beobachteten Spieße lassen sich wie folgt darstellen:

\begin{center}
\begin{tabular}{l|l}
    \textbf{Stände} & \textbf{Früchte} \\
    \hline
    1, 4, 5 & Apfel, Banane, Brombeere \\
    3, 5, 6 & Banane, Pflaume, Weintraube \\
    1, 2, 4 & Apfel, Brombeere, Erdbeere \\
    2, 6 & Erdbeere, Pflaume
\end{tabular}
\end{center}

Nun wird in einer Tabelle jeder Stand mit den Früchten der Spieße, die diesen Stand verwendeten, aufgelistet (eine Spalte pro Spieß).
Zudem werden in einer weiteren Spalte alle Früchte notiert, die in Spießen vorkommen, die nicht diesen Stand verwenden, sie werden verbotene Früchte genannt.
Diese Früchte können nicht von diesem Stand sein.

\begin{center}
\begin{tabularx}{\linewidth}{l|X|X|X}
    \textbf{Stand} & \textbf{Spieße von diesem Stand} & & \textbf{Verbotene Früchte} \\
    \hline
    1 & Apfel, Banane, Brombeere & Apfel, Brombeere, Erdbeere & Banane, Weintraube, Erdbeere, Pflaume \\
    \hline
    2 & Apfel, Brombeere, Erdbeere & Erdbeere, Pflaume & Apfel, Brombeere, Banane, Weintraube, Pflaume \\
    \hline
    3 & Banane, Pflaume, Weintraube & & Banane, Apfel, Brombeere, Erdbeere, Pflaume \\
    \hline
    4 & Apfel, Banane, Brombeere & Apfel, Brombeere, Erdbeere & Banane, Weintraube, Erdbeere, Pflaume \\
    \hline
    5 & Apfel, Banane, Brombeere & Banane, Pflaume, Weintraube & Apfel, Brombeere, Erdbeere, Pflaume \\
    \hline
    6 & Banane, Pflaume, Weintraube & Erdbeere, Pflaume & Banane, Apfel, Brombeere, Erdbeere
\end{tabularx}
\end{center}

Nun werden für jeden Stand alle möglicherweise vorliegenden Früchte gesucht, diese werden im Folgenden legale Früchte genannt.
Alle legalen Früchte für einen bestimmten Stand müssen zwei Regeln erfüllen:
\begin{enumerate}
    \item Die Frucht muss in allen Spießen vorkommen, die diesen Stand benutzen.
    \item Die darf für diesen Stand nicht verboten sein.
\end{enumerate}
Alle legalen Früchte werden nun markiert:

\begin{center}
\begin{tabularx}{\linewidth}{l|X|X|X}
    \textbf{Stand} & \textbf{Spieße von diesem Stand} & & \textbf{Verbotene Früchte} \\
    \hline
    1 & \underline{Apfel}, Banane, \underline{Brombeere} & \underline{Apfel}, \underline{Brombeere}, Erdbeere & Banane, Weintraube, Erdbeere, Pflaume \\
    \hline
    2 & Apfel, Brombeere, \underline{Erdbeere} & \underline{Erdbeere}, Pflaume & Apfel, Brombeere, Banane, Weintraube, Pflaume \\
    \hline
    3 & Banane, Pflaume, \underline{Weintraube} & & Banane, Apfel, Brombeere, Erdbeere, Pflaume \\
    \hline
    4 & \underline{Apfel}, Banane, \underline{Brombeere} & \underline{Apfel}, \underline{Brombeere}, Erdbeere & Banane, Weintraube, Erdbeere, Pflaume \\
    \hline
    5 & Apfel, \underline{Banane}, Brombeere & \underline{Banane}, Pflaume, Weintraube & Apfel, Brombeere, Erdbeere, Pflaume \\
    \hline
    6 & Banane, \underline{Pflaume}, Weintraube & Erdbeere, \underline{Pflaume} & Banane, Apfel, Brombeere, Erdbeere
\end{tabularx}
\end{center}

Die folgenden Früchte sind (von Donald) gefordert: Weintraube, Brombeere und Apfel

Es werden alle Stände durchgegangen, wenn alle legalen Früchte eines Standes gefordert sind, ist dieser Stand ein ausgewählter Stand.
% todo: add reasoning
Wenn keine der legalen Früchte gefordert sind, ist dies kein ausgewählter Stand.
Wenn einige, aber nicht alle legale Früchte gefordert sind, ist nicht genügend Information vorhanden, da nicht gesagt werden kann, ob der Stand eine ausgewählte Frucht enthält oder nicht.

\section{Umsetzung}
% Hier wird kurz erläutert, wie die Lösungsidee im Programm tatsächlich umgesetzt wurde. Hier können auch Implementierungsdetails erwähnt werden.

\section{Beispiele}
% Genügend Beispiele einbinden! Die Beispiele von der BwInf-Webseite sollten hier diskutiert werden, aber auch eigene Beispiele sind sehr gut – besonders wenn sie Spezialfälle abdecken. Aber bitte nicht 30 Seiten Programmausgabe hier einfügen!

\section{Quellcode}
% Unwichtige Teile des Programms sollen hier nicht abgedruckt werden. Dieser Teil sollte nicht mehr als 2–3 Seiten umfassen, maximal 10.

\end{document}
